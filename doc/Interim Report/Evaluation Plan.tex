\section{Evaluation Plan}
To evaluate the performance of the system, several metrics could be used.

For the evaluation of the entire system, one could measure the system's performance in terms of the number of components sorted per minute and compare
this to the number of components that a human can sort per minute. While a crude metric, this would provide a good indication of the system's performance
in terms of speed. Additionally, the percentage of correctly sorted components could be measured, which would indicate the system's accuracy.

For the evaluation of the computer vision system, one could use a multitude of metrics, for example, the accuracy, precision, recall, and F1 score of the
system. These metrics would be calculated by comparing the predictions of the system to the ground truth labels of each component. An important
metric would also be throughput, which would be the number of components that the system can process per minute. This metric would be important
as it would allow us to determine whether the system is fast enough to be used in a real-world setting, in other words, it would be the limiting factor
for the speed of the system.

Alternatively, as the system aims to be a replacement for the current manual sorting process, the system could be evaluated in terms of the time
saved by using the system compared to the time taken to manually sort the components. This would be a more practical metric, as it would allow us to
determine whether the system is a viable replacement for the current manual sorting process. Additionally, as the Lab is working towards the LEAF
certification, the system could be evaluated in terms of the amount of waste it saves compared to just throwing the components away. In the same vein,
the system could be evaluated in terms of the value of the components and the cost of labour it saves.

For the evaluation of the sorting mechanism, the main metric would be the accuracy of the system, which would be the number of components that are
sorted into the correct bin.
