% 3D Printed Frame
\vspace{1em}
\noindent
\textbf{3D Printed Frame} \\
To house the components, a 3D printed housing was designed. It features a modular design, allowing for an iterative design process
where I am able to work on different components of the system independently. It also allows for easy replacement of components
should my design prove to be inadequate. The housing was designed in FreeCAD, a free and open source parametric CAD software 
that I have personal experience with.

The parametric design allows for easy modification of the design should I need to make changes -
with good design practices, I can ensure that changes can be done as I please by simply modifying a few numerical parameters.
The design was then printed using my own personal 3D printer, a heavily modified Voxelab Aquila C2. For this stage on the project,
I must design 3 major components for the housing:
\begin{mylist}
  \item \textbf{PSU Housing} \\
  The PSU housing will contain the power supply and also ensure that all high voltage components are safely enclosed.
  It will also house the power switch, the power socket and the terminals. As I will require step down converters for the Pi and LED strip,
  the PSU housing will also contain the mounting points for the step down converters.
  \item \textbf{Camera Housing} \\
  The camera housing will contain the camera and the LED strip. It will also contain the mounting points for the camera and LED strip.
  The camera housing will also mount an acrylic plate above the camera, so components can be placed on the plate and be imaged by the camera.
  \item \textbf{LCD Display Housing} \\
  The LCD display housing will contain the LCD display and the Raspberry Pi. As the 7" DFRobot LCD display has a Raspberry Pi 4 mount on its back,
  am explicit mount for the Pi is not required. The mount will position the LCD display at an angle, so it can be viewed from above. The LCD display
  simply slides into the mount, allowing for easy removal and also contains mounting points to secure the LCD display to the camera housing.
\end{mylist}
The components are secured using brass M3 heat-set inserts and M3 screws. The heat-set inserts are inserted into the 3D printed parts using a soldering iron,
and the components are then screwed into the inserts. This allows for easy removal of components should I need to make changes to the design, following
my modular design principle.

