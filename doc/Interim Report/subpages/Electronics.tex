\subsubsection{Electronics}
As the project is a physical system, several electronic components are required to make the system work.

\noindent
\textbf{Power Supply} \\
As mentioned in Sections \ref{sec:hardware} and \ref{sec:mechanics}, the system is powered by a 24V, 6.25A, 150W power supply, and must
be stepped down to 5V for the Pi and 12V for the LED Ring. This is achieved using an XL4015 Step Down Converter\cite{xl4015}, a variable
step-down converter that can output up to 75W.
The high power rating of the step-down converter allows for a large overhead, which is useful for future stages of the project where 
additional components may be added to the system.

\noindent
\textbf{Raspberry Pi 4} \\
To power the Raspberry Pi, a standard USB-A to USB-C cable was chosen and was connected to a female USB-A breakout board connected to the
5V step-down converter. This allows for easy removal of the Pi should changes be made to the system, and prevents the unnecessary
cutting of cables.

As the Pi is mounted to the LCD, a USB-A to Micro USB cable is used to connect the Pi to the display, which is then connected to the
Pi's HDMI port, using a supplied HDMI adapter board.

\noindent
\textbf{LED Lighting} \\
As mentioned in Section \ref{sec:hardware}, initially, a 12V LED Light Ring was used and was changed to a WS2812B 5V LED strip, for issues described 
later. To enable the ability to dim the LED Ring, a MOSFET was used to control the LED Ring. As shown in Figure \ref{fig:wiringschematic}, the LED Ring was connected to a 12V supply (achieved using the 12V step-down converter) and
parallel to a 100$\mu$F capacitor, which is used to smooth out the voltage spikes.

The LED Ring was then connected to an N-channel MOSFET IRLZ44N\cite{irlz44n}, which is controlled by a PWM-enabled GPIO pin on the Pi.
This GPIO pin is connected to a 1k$\Omega$ resistor, which prevents potential overcurrent from damaging the Pi. The MOSFET is then connected to ground, completing the circuit.

The IRFZ44N is a logic level MOSFET, meaning it can be fully turned on with a 3.3V gate voltage, which is the voltage of the Pi's GPIO pins, and the high
switching speed (up to 1 mHz) of the MOSFET allows for the LED Ring to be PWM controlled, allowing for the brightness of the LED Ring to be controlled.