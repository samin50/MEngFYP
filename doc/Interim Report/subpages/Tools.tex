\subsection{Tools}
\label{sec:tools}
To streamline the process of developing the system, I developed tools to aid in the development process of the Computer Vision system.

A Jupyter Notebook\cite{jupyter} was developed to aid in developing the Computer Vision system. This is common practice as it
allows code to be run and the data associated with it to be inspected, which is very helpful for debugging purposes. It 
has the capability of training every model that will be used in the Computer Vision system, and contains features like
data augmentation, checkpointing, and model evaluation.

The data augmentation feature is very useful as it allows me to 
retain the original data should I need to add more labels or modify the data in any way. The checkpointing feature allows
me to save the model at any point during training, which serves as a backup but also allows me to resume training and helps 
to protect against overfitting. The model evaluation feature allows me to evaluate the model on a test set, and provides
useful metrics like precision, recall, and mAP. This is useful for evaluating the model's performance, and also for
comparing the performance of different models.

Additionally, a customtkinter\cite{customtkinter} script was developed to aid in the development of Computer Vision system.
This script works in conjuction with the UI, seen in Figure \ref*{fig:mainui}, the Raspberry Pi (made possible using \citet{realvnc})
and allows me to label images with bounding boxes and class labels.

The script has support for multiple components, including:
\begin{multicols}{2}
    \begin{mylist}
        \item Resistors
        \item Capacitors
        \item Ceramic Capacitors
        \item Inductors
        \item Diodes
        \item MOSFETs
        \item Transistors
        \item LEDs
        \item Wires
        \item ICs
    \end{mylist}
\end{multicols}

The script makes use of shortcut keys to quickly label images, and also allows me to save the labels in a format that is
compatible with the YOLOv8 model. Given that this is my own script, I can easily add or remove features as needed - it is likely
that I will need to add more features such as component orientation, so having my own script is very useful.

A screenshot of the script can be seen in Figure \ref*{fig:customtool}.
