\section{Evaluation}
\label{sec:evaluation}
It is important to evaluate the system to determine if it meets the requirements set out in \autoref{sec:project-specification} from a high level, and to determine if the individual components meet their requirements. 

\subsection{Functional Requirements}
This system must be able to meet the requirements set in \autoref{sec:project-specification}. This chapter will evaluate the system against these requirements.

\subsubsection{Mechanical System}
As discussed in \autoref{sec:mechanical-design-evaluation}, the mechanical design enables the system to move components into designated bins for sorting. The system is able to move components from the conveyor belt into the bins, and the bins are able to be removed and replaced with ease. It is also easily maintained and cleaned. Standard parts were used so the system remains cost-effective and easy to repair. 

The main mechanical system is the conveyor belt, and the sweeper and bin system. The conveyor's function requirements outlined in \autoref{sec:conveyor-design} were met, as the conveyor is able to move components from the input to the sweeper, and provide mechanisms for tensioning and detaching the conveyor belt. It allows the rollers to turn freely though the use of bearings, facilitating smooth operation.

Likewise, the sweeper and bin system design allows the system to move components from the conveyor to the bins, and the bins are able to be removed and replaced with ease. The sweeper is able to move components off the conveyor belt and into the bins, meeting the functional requirements of needing a sorting system for the components.

However, it is important to note that the system can only sort one component at a time, which is a limitation of the system. This is due to the fact that the sorting arm requires the movement of the conveyor belt to move the components off the belt, rather than removing components from the conveyor belt directly. This is a limitation of the system but a difficult problem to solve, as the system would require a more complex mechanism to remove components from the conveyor belt directly, such as a vacuum system which may not be feasible, as outlind in \autoref{sec:sweeper-design}.

\subsubsection{Electronics and Wiring}


\subsection{Non-Functional Requirements}
The system has achieved its goal of helping to being the Level 1 Electrical Engineering Labs closer to its LEAF certification by providing a way to reduce the amount of waste produced by the labs. 