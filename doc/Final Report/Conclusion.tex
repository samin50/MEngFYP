\section{Conclusion}
\label{sec:conclusion}

This project was an incredibly complex and challenging project that required a lot of research and development across several engineering disciplines to complete. The project required extensive programming skill to complete to ensure system stability and coherency, as well as advanced CAD skills to be able to fully realise the physical parts of the system. The project also demanded knowledge of deep-learning to properly identify the appropriate model for the task of component identification, as well as sufficient knowledge of electronics to be able to design and implement the electronics system. However, all of these skills were developed and honed throughout the project, and the project was completed largely successfully, but not without its challenges and it is not without its limitations.

\subsection{Future Works}
Due to time constraints, there were some features that were not implemented in the final system, but could be implemented in future iterations of the system. One of these features is a feeder system as described in \autoref{sec:background} in order to autonomously feed the components into the system one by one for sorting. This is an incredibly difficult design and engineering challenge, as it requires a system that can successfully detangle and feed the components into the system, and this was not feasible to implement in the time frame of this project.

Another improvement to the system would be to reduce the inference latency on the trained models by defining custom recipes for the SparseML and DeepSparse libraries, as discussed in \autoref{sec:sparsification-deployment}, to reduce the model size and increase the inference speed. This would allow for the system to be able to process components faster, and thus sort them faster. Alternatively, an improvement could instead be to use a more powerful device to run the models, such as a Jetson Nano as explored in \autoref{sec:design-hardware}, or even offload the inference to a cloud service, which would allow for the system to be able to process components faster, provided that the internet connection is stable.

Additionally, to perform value identification, deep-learning models capable of OCR could be used to read the values of the components, as discussed in \autoref{sec:component-value-identification}, and the resistor model should be retrained with a higher resolution dataset to improve the accuracy of the model. Finally, the component identification model should be trained with more component classes to improve its breadth of identification.
